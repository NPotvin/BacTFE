% !TeX program = pdflatex
% !TeX encoding = UTF-8
% !TeX spellcheck = fr
% !TeX TXS-program:bibliography = txs:///biber
%
% Template d'article avec programmes (en français) - Version PDFLaTeX
%
% Yves ROGGEMAN - octobre 2017 - PDFLaTeX
%
\documentclass[a4paper,twoside,10pt,english,french]{article}   % ,twocolumn
%---                           % Encodage pour PDFLaTeX  
\usepackage[T1]{fontenc}       % Encodage le plus étendu
\usepackage[utf8]{inputenc}    % Source Unicode en UTF-8
\usepackage{comment}           % Commentaires multilignes
%---
\title{À la tâche : Lorem ipsum cum computatione}
\author{Nicolas Potvin~\thanks\thethanks}
\date{\DTMToday}
\newcommand{\thethanks}{Université libre de Bruxelles (ULB)
<\href{mailto:nicolas.potvin@ulb.ac.be}{nicolas.potvin@ulb.ac.be}>}
\newcommand{\thesubject}{Canevas LaTeX avec algorithme}
\newcommand{\thekeywords}{LaTeX ; PDF ; algorithme}
\newcommand{\thelang}{fr}
% Définit les variables du titre (pour usage dans le document)
\makeatletter
\let\thetitle\@title
\let\thedate\@date
{ % Définit '\theauthor' sans '\thanks' et '\thefilename' sans "..."
\def\auth@f#1~#2@{#1}\xdef\theauthor{\expandafter\auth@f\@author~@}
\def\fn@f#1.#2@{#1}\xdef\thefilename{\expandafter\fn@f\jobname.@.tex}
}
\makeatother
% ---------------------------------------------------------------------------
% Mise en page (voir aussi geometry)
% \setlength{\parindent}{0pt}
\setlength{\parskip}{\medskipamount}
\setlength{\textfloatsep}{\intextsep}
\renewcommand{\topfraction}{1}
\renewcommand{\textfraction}{0}
% ---------------------------------------------------------------------------
% Paragraphe et note en marge
\newcommand{\newpar}{\par\addvspace{\medskipamount}}
\reversemarginpar
\newcommand{\mymarginpar}[1]{%   Colle au texte !
  \marginpar[\raggedleft\footnotesize#1]{\raggedright\footnotesize#1}
}
% ---------------------------------------------------------------------------
% Packages de base
\usepackage{etoolbox}          % Utilitaires e-TeX (tests, variables, flags, etc.)
\usepackage{xkeyval}           % Syntaxe des paramètres [par=...]
\usepackage{xspace}            % Gestion intelligente des espaces après macros
%---                           % Polices pour PDFLaTeX
\usepackage{babel}             % En français, s.v.p. (avec anglais 2d)
\frenchbsetup{StandardItemizeEnv=true,StandardItemLabels=true} % Paramètres pour frenchb.dtx
\usepackage[full]{textcomp}    % Jeu de symboles du "Companion"
\usepackage{amsmath}           % Commandes supplémentaires en mode "math"
\usepackage{amssymb}           % Jeu de symboles AMS, sur-ensemble d'amsfonts
\usepackage{amsthm}            % Environnement "theorem" enrichi
\usepackage{mathcomp}          % Textcomp en mode math
\usepackage{mathdots}          % Redéfinit les \x_dots et ajoute \iddots
\usepackage{mathtools}         % Compléments et corrections d'amsmath
\usepackage{textgreek}         % Caractères grecs (hors math)
\usepackage{csquotes}
%---
\usepackage[frenchstyle]{kpfonts}   % Polices "Kp-Fonts", alias Kelpler (GPL) -- option : easyscsl,
% \usepackage{dejavu}            % Polices DéjàVu - Attention : sans SmallCaps
\usepackage{DejaVuSerif}
\usepackage[scaled]{DejaVuSans}
\usepackage[scaled]{DejaVuSansMono}
%---
\renewcommand{\familydefault}{\sfdefault}  % Pour polices sans empattement
%---
% \ifdefstring{\familydefault}{\sfdefault}{\mathversion{sf}}{}   % math sf aussi (sfmath in kpfonts)
\usepackage[frenchmath]{mathastext} % ,unicodeminus
%
\let\oldscshape=\scshape
\ifdefstring{\familydefault}{\sfdefault}     % SmallCaps de Kp-Fonts (si nécesaire)
{\renewcommand*{\scshape}{\fontfamily{jkpssk}\oldscshape\selectfont}}%
{\renewcommand*{\scshape}{\fontfamily{jkpk}\oldscshape\selectfont}}%
%---
\usepackage{geometry}          % Dimension des marges etc.
\geometry{                     % Paramètres pour dito
  inner=25mm,outer=20mm,
  top=25mm,bottom=25mm,head=5mm,
  marginpar=15mm,marginparsep=6pt,
}
\usepackage{enumitem}          % Mise en page des listes
\setlist[1]{labelindent=\parindent} % paramètres pour dito (identation de base)
\usepackage{graphicx}          % Inclusion de figures et graphiques
\graphicspath{{$HOME/Images/Logos/}{$HOME/}{../../common_img/}}
\usepackage{xcolor}            % Caractères colorés
\definecolor{Pantone287}{RGB}{0,76,147} % Couleur Logo ULB
\usepackage{fancyvrb}          % Environnement 'Verbatim' (avec majuscule !)
\usepackage{numprint}          % Formatage des nombres
\npfourdigitnosep              % Paramètres pour dito
\usepackage{caption}           % Titres des figures et tableaux plus corrects
\captionsetup{labelfont=sc,labelsep=endash,position=top} % Paramètres pour dito
\captionsetup[figure]{name=Schéma}                       % idem (suite)
\captionsetup[table]{name=Table}                         % idem (suite)
% \usepackage{tabularx}          % Tableaux à colonnes de largeur variable
\usepackage{longtable}         % Tableaux sur plusieurs pages etc.
\usepackage[                   % Insertion de marques "to do"
  textsize=tiny,
  textwidth=13mm,
  backgroundcolor=yellow
  ]{todonotes}
\usepackage{hyperref}          % Table des matières en signets (PDF-LaTeX)
\hypersetup{                   % Paramètres pour dito
  unicode,
  bookmarksnumbered,
  breaklinks,
  nesting,
  pdftitle={\thetitle},
  pdfauthor={\theauthor},
  pdfsubject={\thesubject},
  pdfkeywords={\thekeywords},
  pdflang={\thelang},
}
%---
% \VerbatimFootnotes             % Permet ainsi \verb en note de bas de page
% ---------------------------------------------------------------------------
% Outils graphiques ("Portable Graphics Format" + "TikZ ist kein Zeichenprogramm")
\usepackage{pgf}               % in tikz, mais utile pour option
\usepgflibrary{decorations.markings}
\usepackage{tikz}              % in pgfplots, mais utle pour biblios
\usetikzlibrary{shapes,arrows,chains,calc}
\usepackage{pgfplots}          % Graphiques en 2D et 3D
\pgfplotsset{compat=1.15}      % Bonne version pour dito
% \usetikzlibrary{               % autres options ?
%   backgrounds,
%   mindmap,
%   trees,
%   automata,
%   positioning,
%   fit,
%   shapes.multipart
% }
% ---------------------------------------------------------------------------
% En-tête et pied de page
\usepackage[useregional,showdow,showseconds=false,showzone=false,%
    datesep=/,timesep=\ensuremath{\!\colon\!\!}]{datetime2}
\usepackage{lastpage}
\usepackage{fancyhdr}
\pagestyle{fancy}
  \renewcommand{\sectionmark}[1]{\markboth{#1}{}}
  \renewcommand{\subsectionmark}[1]{\markright{#1}}
  \fancyhf{}
  \fancyhead[LO,RE]{\nouppercase{\leftmark}}
  \fancyhead[RO,LE]{\slshape\footnotesize\nouppercase{\rightmark}}
  \fancyfoot[LO,RE]{\footnotesize\texttt{\thefilename} \\ \DTMsetstyle{default}\DTMnow}
  \fancyfoot[C]{-~\thepage~/~\pageref{LastPage}~-}
  \fancyfoot[RO,LE]{\footnotesize\theauthor \\ \thetitle}
%
\fancypagestyle{plain}{ %  Première page ----------------------
  \fancyhf{}
  \renewcommand{\headrulewidth}{0pt}
  \fancyheadoffset[R]{15mm}
%   \addtolength{\headheight}{35pt}    % pour warning, mais...
%   \addtolength{\textheight}{-35pt}   % semble sans effet !
  \fancyhead[L]{
    \raisebox{-7mm}{
      \parbox{\textwidth}{
        \includegraphics{barrette-original} \\[\baselineskip]
        \fontsize{8pt}{10pt}\selectfont
        \sffamily\color{Pantone287}
        FACULTÉ DES SCIENCES       \\
        DÉPARTEMENT D'INFORMATIQUE
      }
    }
  }
  \fancyhead[R]{
    \raisebox{-10mm}[0pt][0pt]{\includegraphics[width=120mm]{ULB-ligne-gauche}}
  }
  \fancyfoot[L]{\raisebox{0mm}{}\color{Pantone287}\footnotesize\texttt{\thefilename} \\
    \DTMsetstyle{default}\DTMnow}
  \fancyfoot[C]{-~\thepage~/~\pageref{LastPage}~-}
  \fancyfoot[R]{
    \raisebox{-12pt}{\includegraphics[height=\footskip]{sceau-b-quadri}}
  }
} % Fin de première page
% ---------------------------------------------------------------------------
% Mise en page de programmes et algorithmes
\usepackage{listings}
\lstdefinelanguage[Extended]{C++}[ISO]{C++}{ % extension: description d'algo
  morekeywords={algo,assert,downto,esac,fi,forall,input,loop,mod,od,
                otherwise,output,pool,repeat,stopif,then,until,upto,when},
  literate={<-}{\ensuremath{\leftarrow}}{3}
           {<->}{\ensuremath{\rightleftharpoons}}{3} % after '<-' definition
           {&&}{\ensuremath{\wedge}}{3}
           {||}{\ensuremath{\vee}}{3}
           {!}{\ensuremath{\neg}}{3}
           {<=}{\ensuremath{\leq}}{3}
           {>=}{\ensuremath{\geq}}{3}
           {!=}{\ensuremath{\neq}}{3} % after '!' definition
           {==}{\ensuremath{\equiv}}{3},
}
\lstloadlanguages{[ISO]C++,[Extended]C++}
% Styles définis pour environnement {lstlisting}
\lstdefinestyle{Clong}{ % Pour inclusion fixe d'un fichier
  firstline=5,
  firstnumber=1,
  belowskip=\bigskipamount,
  frame=tb,
  columns=[c]flexible,
}
\lstdefinestyle{Cfloat}{ % Pour inclusion flottante d'un fichier
  firstline=5,
  firstnumber=1,
  float,
  frame=tb,
  columns=[c]flexible,
}
\lstdefinestyle{Cpart}{ % Pour morceaux de programme
  texcl,
  escapeinside={/*@}{@*/},
  escapebegin={\sffamily\itshape},
  columns=[c]flexible,
}
\lstdefinestyle{Cin}{ % Pour écriture directe d'un code C++
  frame=tb,
  texcl,
  escapeinside={/*@}{@*/},
  escapebegin={\sffamily\itshape},
  columns=[c]flexible,
}
\lstdefinestyle{Cext}{ % Pour description d'un algorithme
  language=[Extended]C++,
  texcl,
  mathescape,
  escapeinside={/*@}{@*/},
  escapebegin={\sffamily\itshape},
  columns=[c]flexible,
}
\lstset{ % Paramètres par défaut définis en fonction de \lstinline
  language=[ISO]C++,
  gobble=2, % Oubli des 2 premiers caractères de la ligne
  basicstyle={\ttfamily\small},
  commentstyle={\sffamily\itshape},
%   stringstyle=\itshape,
  tabsize=3,
  numbers=left,
%   stepnumber=5,
  numberstyle=\tiny,
  numbersep=6pt,
  floatplacement=htb,
  belowcaptionskip=\medskipamount, % \bigskipamount,
  xleftmargin=28.45275pt, % = 1cm
  framexleftmargin=28.45275pt, % = 1cm
%   resetmargins,
  breaklines,
  breakindent=28.45275pt, % = 1cm
  columns=[c]fullflexible,
  keepspaces=true,
  fancyvrb=true,
}
% Commandes associées utiles
% \newcommand{\lstlbl}[1]{#1} % Pour étiqueter un programme (cf. \label)
\newcommand{\lstref}[1]{ % Pour référence en marge à un fichier de programmes
  \mymarginpar{\tiny\ttfamily<#1>}
}
\newcommand{\lstinputf}[3]{ % Inclusion flottante
  \mymarginpar{\tiny\ttfamily<#1>}
  \lstinputlisting[style=Cfloat,caption={#2 - \texttt{<#1>}},label={#3}]{#1}%
}
\newcommand{\lstinput}[3]{ % Inclusion fixe
  \mymarginpar{\tiny\ttfamily<#1>}
  \lstinputlisting[style=Clong,caption={#2 - \texttt{<#1>}},label={#3}]{#1}
}
\newcommand{\lstfn}{\lstinline [basicstyle=\footnotesize]} % En bas de page
% ---------------------------------------------------------------------------
% Abréviations utiles
\newcommand{\cpp}{\mymarginpar{\fbox{\large\ttfamily C++}}}
\newcommand{\Cpp}{\texttt{C++}\xspace}
\newcommand{\Java}{\texttt{Java}\xspace}
\newcommand{\ie}{\textit{i.e.}\ }  % ID EST
\newcommand{\eg}{\textit{e.g.}\ }  % EXEMPLI GRATIA
\newcommand{\cf}{\textit{cf.}\ }   % CONFER
\newcommand{\vs}{\textit{vs.}\ }   % VERSUS
\newcommand{\PP}{\mathbb{P}}       % Probability: \Pr variant
\newcommand{\Exp}{\mathbb{E}}      % Expected value
\DeclareMathOperator{\rank}{rank}  % Rank of matrix
\DeclareMathOperator{\im}{im}      % Image of
\DeclareMathOperator{\lcm}{lcm}    % Least Commun Mutliple
% ---------------------------------------------------------------------------
% Environnements "théorème" francisés
% \newcommand{\qed}{\hspace*{1em}\hfill\ensuremath{\blacksquare}}  % QUOD ERAT DEMONSTANDUM (U+220E)
\renewcommand{\qedsymbol}{\ensuremath{\blacksquare}}  % QUOD ERAT DEMONSTANDUM (U+220E)
\theoremstyle{definition}
\newtheorem*{definition}{Définition}
\theoremstyle{remark}
\newtheorem{remarque}{Remarque}
\newtheorem*{exemple}{Exemple}
\newtheorem*{note}{Note}
\theoremstyle{plain}
\newtheorem{theoreme}{Théorème}[section]
\newtheorem{lemme}[theoreme]{Lemme}
\newtheorem{corollaire}[theoreme]{Corollaire}
\newtheorem{propriete}[theoreme]{Propriété}
\newtheorem{conjecture}[theoreme]{Conjecture}
% ---------------------------------------------------------------------------
% Contexte "questions"
\newcounter{quest}
\newcommand{\question}[2]{% En-tête "Question"
  \addtocounter{quest}{1}
  \section*{Question~\thequest.~-- #1 {\small(#2~points)}}%
}
% ---------------------------------------------------------------------------
\begin{document}
%
% Titre des preuves en gras
\let\oldproofname=\proofname
\renewcommand{\proofname}{\bfseries\oldproofname}
% Version française des titres (surcharge babel après \begin{document})
\renewcommand{\listfigurename}{Liste des schémas}
\renewcommand{\listtablename}{Liste des tables}
\renewcommand{\lstlistingname}{Programme}
\renewcommand{\lstlistlistingname}{Liste des programmes}
%
\maketitle
%
\begin{abstract}
Lorem ipsum dolor sit amet, consectetur adipisicing elit, sed do eiusmod tempor incididunt ut labore et dolore magna aliqua. Ut enim ad minim veniam, quis nostrud exercitation ullamco laboris nisi ut aliquip ex ea commodo consequat.
\end{abstract}

%\setquotestyle{english}
\textquote[Georges Perec]{Portez ce vieux whisky au juge blond qui fume.}
%\setquotestyle{french}
\blockquote[Bel inconnu]{\sl Portez ce vieux whisky au juge blond qui fume sur son île intérieure, à côté de l’alcôve ovoïde, où les bûches se consument dans l’âtre, ce qui lui permet de penser à la cænogénèse de l’être dont il est question dans la cause ambiguë entendue à Moÿ, dans un capharnaüm qui, pense-t-il, diminue çà et là la qualité de son œuvre.}
%\setquotestyle{french}

\section{On teste}

Dès Noël où un zéphyr haï me vêt\textbf{[ le flanc]} de glaçons würmiens,\textit{[ en effet,]} je dîne enfin d'exquis rôtis de bœuf au kir à l'aÿ d'âge mûr \& cætera. (Gilles Esposito-Farèse)

\textrm{Dès Noël où un zéphyr haï me vêt\textbf{[ le flanc]} de glaçons würmiens,\textit{[ en effet,]} je dîne enfin d'exquis rôtis de bœuf au kir à l'aÿ d'âge mûr \& cætera.}

M\textsc{Mx}x. \textsc{Dès Noël où un zéphyr haï me vêt\textbf{[ le flanc]} de glaçons würmiens,\textit{[ en effet,]} je dîne enfin d'exquis rôtis de bœuf au kir à l'aÿ d'âge mûr \& cætera.}

DÈS NOËL OÙ UN ZÉPHYR HAÏ ME VÊT\textbf{[ LE FLANC]} DE GLAÇONS WÜRMIENS,\textit{[ EN EFFET,]} JE DÎNE ENFIN D'EXQUIS RÔTIS DE BŒUF AU KIR À L'AŸ D'ÂGE MÛR \& CÆTERA.

\noindent L'alphabet français comprend aujourd'hui 42~lettres (et non 26): \\* % Warning ?
\indent a b c d e f g h i j k l m n o p q r s t u v w x y z
\hspace*{1cm} à â é è ê ë î ï ô ù û ü ÿ ç æ œ \\
ainsi que les ligatures techniques: \\*
\indent ff fi fl ffi ffl   \\
% \indent ff  fi  fl  ffi   ffl  \\
et parfois les ligatures antiques: \\*
\indent % ſt
st sp ct cp    \\
% \indent ſt  st

Les naïfs ægithales hâtifs pondant à Noël dans un cañon où il gèle sont sûrs d'être déçus et de voir leurs drôles d'œufs abîmés. À quoi bon? Ça ne marche pas. Évidemment! Œdipe et Ève avaient raison. Ô Ânes bâtés, demandez-vous: xxxxxxxxxxxxxxxxxxx \og Être ou ne pas être?\fg ou \enquote{Être ou ne pas être?}

\subsection{On reprend}

Les naïfs ægithales hâtifs pondant à Noël dans un cañon où il gèle sont sûrs d'être déçus et de voir leurs drôles d'œufs abîmés. À quoi bon? Ça ne marche pas. Évidemment! Œdipe et Ève avaient raison. Ô Ânes bâtés, demandez-vous: \enquote*{Être ou ne pas être?} ou \enquote{Être ou ne pas être?}, entre nous.

\section{Scientia gratias}

Lorem ipsum dolor sit amet, consectetuer adipiscing elit. Morbi sed justo. Cras venenatis rhoncus eros. Ut dictum elit vitae diam. Praesent in velit. \mymarginpar{In illo tempore erat fortissimus vir}Proin aliquam nulla eget nisl. Donec sed diam ac sem hendrerit mollis. Suspendisse id justo. Duis mollis, risus quis gravida dapibus, magna tortor sodales mauris, et ultricies magna orci a eros. Aliquam ac nisi. Aliquam nonummy placerat leo. In gravida. Aliquam erat volutpat. Phasellus euismod mauris non justo. Mauris ante ligula, semper pellentesque, sollicitudin nec, varius at, mauris. Mauris nec metus nec quam egestas congue.

Nam ut lorem sed neque lobortis iaculis. Nunc nulla lorem, dictum non, auctor sit amet, tempus ut, arcu. Vivamus tortor ante, porta a, pellentesque non, semper et, enim.
\begin{itemize}
  \item Phasellus sodales. Aliquam erat volutpat. Suspendisse magna neque, luctus ac, dictum at, sagittis sed, turpis.
  \item Donec tempus dolor vitae nibh aliquet dictum.
  \item Lorem ipsum dolor sit amet, consectetuer adipiscing elit.
\end{itemize}
Pellentesque habitant morbi tristique senectus et netus et malesuada fames ac turpis egestas. Nam tempus sodales sem. Nam risus pede, rutrum sit amet, molestie a, mollis a, nibh. Curabitur lorem ligula, bibendum a, rhoncus at, tempus nec, ante. Nulla quis libero in sapien laoreet fringilla. Aenean justo.

\subsection{Ad vitam}

Nunc tempor. Nam elit. Praesent vulputate. Praesent\footnote{Lorem ipsum dolor sit amet, consectetuer adipiscing elit.} risus. Quisque convallis lectus vitae erat. Integer eget urna eu enim dictum eleifend.
\begin{enumerate}
  \item Fusce malesuada ultricies tortor.
  \item Pellentesque semper pede eu eros.
\end{enumerate}
Quisque in nisl. Ut dapibus leo eget tellus.

Duis ac lacus. Integer dapibus, erat et iaculis pulvinar, mauris leo varius purus, eget interdum massa tellus nec mauris. Phasellus malesuada egestas lorem. \mymarginpar{Nunc est bibendum} Mauris nisi massa, rutrum in, varius non, nonummy et, est. Nunc volutpat. Nullam ultrices aliquam orci. Nullam eget neque. Class aptent taciti sociosqu ad litora torquent per conubia nostra, per inceptos hymenaeos. Nunc aliquam fermentum turpis. Nullam elit felis, nonummy sed, dictum ac, vestibulum ut, augue. Curabitur orci. Maecenas at odio vitae felis congue lobortis. Sed iaculis metus et ante iaculis vestibulum.

In molestie leo sollicitudin pede. Cras porttitor nulla nec nulla. Vivamus cursus mollis purus. Aliquam ac sapien. Nullam metus felis, porta quis, convallis eget, adipiscing quis, dolor. Duis accumsan velit et turpis. Curabitur posuere. Donec vitae magna. Etiam tellus mi, laoreet sit amet, fermentum in, fringilla ut, mi. Ut mi risus, iaculis sit amet, gravida vitae, suscipit et, orci. Duis sit amet enim sed lectus ullamcorper fringilla. Sed quis sapien.

\section{Élève d'Euclide}

Euclide (Ευκλειδης) nous a enseigné une méthode efficace \texttt{\textbf{algo} PGCD} de calcul du plus grand commun diviseur de deux nombres naturels.\cpp
\begin{lstlisting}[style=Cext,caption={Algorithme d'Euclide},label={alg:euclide}]
  algo PGCD (a, b)
    input a, b
    while b != 0 do
      r := a mod b /*@\label{alg:Refmod}@*/
      a := b; b := r
    od
    output a
\end{lstlisting}

\subsection{On va analyser}

Pour toutes valeurs entières non négatives $a$ et $b$ équiprobables ($\sum_{i=1}^n a_{\ell_i} = 1$), le nombre
\emph{moyen} d'itérations du programme~\ref{alg:euclide} est
  \[ X_{i_j}^* = H(b) = \sum_{r=1}^b \frac{1}{r} \qquad \forall\ a, b \]
et asymptotiquement, on a
  \[ H(b) = \ln b + \gamma + \frac{1}{2b} - o(\frac{1}{b}) \]
où \todo{Origine?}$\gamma\approx0.5772156649$ est la \emph{constante d'Euler}.

\begin{table}[htb]
  \caption{Complexité de l'algorithme d'Euclide}
  \centering
  \begin{tabular}{l|ccc}
    Complexité & Minimale         & En moyenne       & Maximale         \\
    \hline
    Spatiale   & $\Theta(1)$      & $\Theta(1)$      & $\Theta(1)$      \\
               & $3$              & $3$              & $3$              \\
    \hline
    Temporelle & $\Theta(1)$      & $\Theta(\log b)$ & $\Theta(\log b)$ \\
               & $0$              & $H(b)$           &
                             $\left\lceil \log_\phi b \right\rceil + 1$ \\
  \end{tabular}
\end{table}

\begin{theoreme}
  Le nombre \emph{maximum} d'itérations est indépendant de $a$ et, pour $b>0$\footnote{Il est nul pour $b=0$.}, il vaut
  \[ \left\lceil \log_\phi b \right\rceil + 1 \]
  où $\phi=1+\frac{1}{\phi}=\frac{1+\sqrt{5}}{2}\approx1.618034$ est le \enquote{nombre d'or}; cette borne est atteinte pour certaines données.
\end{theoreme}

\begin{proof}
  C'est immédiat!
\end{proof}

% ---------------------------------------------------------------------------
%
\clearpage
\fancyhead[RO,LE]{}
\setlength{\parskip}{0pt}
\addcontentsline{toc}{section}{\protect\numberline{}{\listfigurename}}
\listoffigures
\addcontentsline{toc}{section}{\protect\numberline{}{\listtablename}}
\listoftables
\addcontentsline{toc}{section}{\protect\numberline{}{\lstlistlistingname}}
\lstlistoflistings
%
\begin{thebibliography}{99}
  \addcontentsline{toc}{section}{\protect\numberline{}{\refname}}

  \bibitem{SHANNON_WEAVER}
  \textsc{Shannon}, Claude E. \& \textsc{Weaver}, Warren,
  \emph{The Mathematical Theory of Communication},
  Vol\@. 1,
  132~pp.
  The University of Illinois Press, Urbana
  (Sept\@. 1949).

  \bibitem{MASSEY}
    \textsc{Massey}, James Lee,
    \enquote*{Shift-register synthesis and BCH decoding},
    \emph{IEEE Trans\@. Inf\@. Theory},
    \textbf{IT-15}(1),
    122--127
    (Jan\@. 1969).

\end{thebibliography}
%
\addcontentsline{toc}{section}{\protect\numberline{}{\contentsname}}
\tableofcontents
\clearpage
\listoftodos
%
\end{document}
%
